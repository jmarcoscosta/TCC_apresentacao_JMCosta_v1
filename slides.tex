%--------------------------------------------%
% Template Beamer para Apresentações da UFRN %
% by alcemygvseverino@gmail.com              %
% Baseado em MIT Beamer Template			 %
% versao 1.1								 %
% Atualizado em 14/05/2016					 %
%--------------------------------------------%
\documentclass[handout,t]{beamer}
% Para alterar a linguagem do documento
\usepackage[portuges]{babel}
% Para aceitar caracteres especias deretamente do teclado
\usepackage[utf8]{inputenc}
% Para seguir as normas da ABNT de citacao e referencias
\usepackage[alf]{abntex2cite}
% Para incluir figuras
\usepackage{graphicx}
% Para melhor ajuste da posisao das figuras
\usepackage{float}
% Para ajustar as dimensoes do layout da pagina
\usepackage{geometry}
% Para formatar estrutura e informacoes de formulas matematicas
\usepackage{amsmath}
% Para incluir simbolos especiais em formulas matematicas
\usepackage{amssymb}
% Para incluir links nas referencias
\usepackage{url}
% Para incluir paginas de documentos .pdf externos
\usepackage{pgfpages}
% Para ajustar o estilo dos contadores
\usepackage{enumerate}
% Para modificar a cor do texto
\usepackage{color}
% Para incluir condicoes
\usepackage{ifthen}
% Para colocar legendas em algo que nao e float
\usepackage{capt-of}
% Para definir o tema do slide
\usetheme{Berlin}
% Para difinir cores e background
\usecolortheme{ufrn}
% Para numerar as figuras
\setbeamertemplate{caption}[numbered]

% Título
\title[Trabalho de conclusão de curso]{
	Implementação do sistema de arquivos SquashFS no bootloader U-Boot}
% Data
\date{
	\today}
% Autores
\author[João Marcos Correia de Lima Costa]{
	João Marcos Correia de Lima Costa %\inst{1}\\
	%\vspace{0.25cm}
	%Autor 02 \inst{2}}
	}
% Instituto
\institute[INSTITUTO]{
	%\inst{1}%
	%\url{jmcosta944@mail.com}\\
	\vspace{0.25cm}
	%\inst{2}%
	Departamento de Engenharia Elétrica\\
	Universidade Federal do Rio Grande do Norte}
% Logo do canto inferior direito
\pgfdeclareimage[height=0.7cm]{logo_UFRN}{figuras/logo_UFRN}
\logo{
	\vspace*{-0.25cm}
	\pgfuseimage{logo_UFRN}
	\hspace*{-0.05cm}}


\begin{document}
% Sumário
\frame{\titlepage}
\section[]{}
\begin{frame}{Sumário}
	\tableofcontents
\end{frame}

% Introducao

\section{Introdução}
\begin{frame}{Introdução}
	\begin{itemize}
	    \item Trabalho desenvolvido durante o intercâmbio internacional (BRAFITEC)
	    \item Tema do estágio profissional do último ano de curso da École nationale supérieure d'ingénieurs de Caen (ENSICAEN)
	    \item Estágio profissional sediado na Bootlin, empresa francesa especializada em Linux embarcado, localizada em Toulouse, França.
	\end{itemize}
\end{frame}

\begin{frame}{ENSICAEN}

\begin{itemize}
    \item École nationale supérieure d'ingénieurs de Caen
    \item Situada em Caen (França), na região da Normandia
    \item Dois anos de intercâmbio, na especialidade de Eletrônica e Física aplicada
\end{itemize}

\begin{figure}
    \centering
    \includegraphics[scale=0.1]{figuras/ENSICAEN-batiment-A.jpg}
    \includegraphics[scale=0.1]{figuras/ENSICAEN-batiment-A.jpg}
    %\caption{Caption}
    \label{fig:my_label}
\end{figure}
    
\end{frame}

\begin{frame}{Bootlin}

    \begin{itemize}
        \item Empresa prestadora de serviços e treinamentos em Linux embarcado
        \item Forte presença no mundo do software livre
        \item Escritórios em Toulouse, Lyon e Orange
    \end{itemize}
    
    \begin{figure}
        \centering
        \includegraphics[scale=0.6]{figuras/bootlin_logo.png}
        %\caption{Logo da empresa}
        \label{fig:my_label}
    \end{figure}

\end{frame}





% Metodologia
\section{Metodologia}
\begin{frame}{Metodologia}
	\begin{enumerate}
	    \item Desenvolver uma ferramenta em linha de comando para ser usado em \textit{user-space}
	    \item Migrar o código para o U-Boot
	    \begin{itemize}
	        \item Implementar a API de sistemas de arquivo do U-Boot
	        \item Implementar os comandos do U-Boot
	    \end{itemize}
	    \item Submeter o trabalho na \textit{mailing list} do U-Boot
	\end{enumerate}
\end{frame} 

\begin{frame}{Ferramenta de linha de comando: \textit{squashfs-utils}}
	\begin{itemize}
	    \item Ferramenta: \textit{squashfs-utils}
	    \item Construída \textit{from scratch}, usando uma documentação não-oficial do SquashFS
	    \item Entender como o sistema de arquivos se organiza
	    \item Definir estruturas de dados essenciais
	    \item Analisar e imprimir as seções da imagem SquashFS
	\end{itemize}
\end{frame}

\begin{frame}{Gerando uma imagem SquashFS}
    \begin{figure}
        \centering
        \includegraphics[scale=0.25]{figuras/mksquashfs.png}
        \caption{Compilação de uma imagem SquashFS}
        \label{fig:my_label}
    \end{figure}
\end{frame}

\begin{frame}{Imagem SquashFS}
    \begin{figure}
        \centering
        \includegraphics[scale=0.32]{figuras/sqfs.png}
        \caption{Layout de uma imagem SquashFS}
        \label{fig:my_label}
    \end{figure}
\end{frame}

\begin{frame}{Ferramenta de linha de comando: \textit{squashfs-utils}}
    \begin{figure}
        \centering
        \includegraphics[scale=0.4]{figuras/sqfsutils.pdf}
        \caption{Uso da \textit{squashfs-utils}}
        \label{fig:my_label}
    \end{figure}
\end{frame}

\begin{frame}{Ferramenta de linha de comando: \textit{squashfs-utils}}
Vamos executar alguns comandos para analisar uma imagem SquashFS gerada a partir do diretório abaixo:
\begin{figure}
    \centering
    \includegraphics[scale=0.7]{figuras/tree.pdf}
    \caption{Diretório fonte: diretório vazio, arquivo de texto e link simbólico}
    \label{fig:my_label}
\end{figure}
\end{frame}


\begin{frame}[fragile]{Ferramenta de linha de comando: \textit{squashfs-utils}}

Instruções de uso:

\begin{minted}[fontsize=\fontsize{8}{8}, bgcolor=blcodebg]{text}
$ ./sqfs -h
usage: sqfs [-h]
       sqfs [-s] [-i] [-d] <fs-image>
       sqfs [-e] <fs-image> /path/to/dir/
       sqfs [-e] <fs-image> /path/to/file

Tool to analyze the content of a SquashFS image

Options:
       -h: Prints the usage and exits
       -s: Dumps the contents of a SquashFS image's superblock
       -i: Dumps the contents of a SquashFS image's inode table
       -d: Dumps the contents of a SquashFS image's directory table
       -e: Dumps the contents of a SquashFS image's file or directory.
       For directories, end path with '/'.

Parameters:
       <fs-image>: Path to the filesystem image
\end{minted}    
\end{frame}


\begin{frame}[fragile]{Ferramenta de linha de comando: \textit{squashfs-utils}}
Informações do \textit{Superblock}:
\begin{minted}[fontsize=\fontsize{7}{7}, bgcolor=blcodebg]{text}
$./sqfs -s source-dir.sqfs 
--- SUPER BLOCK INFORMATION ---
Magic number: sqsh
Number of inodes: 4
Filesystem creation date: Tue 2020-08-04
(yyyy-mm-dd) 15:46:14 CEST
Block size: 131 kB
Number of fragments: 1
Block log: 17
Compression type: ZLIB
Super Block Flags: 0xc0
Major/Minor numbers: 4/0
Root inode: 0x60
Bytes used: 312
Id table start: 0x130
(xattr) Id table start: 0xffffffffffffffff
Inode table start: 0x6c
Directory table start: 0xbc
Fragment table start: 0x105
Lookup table start: 0x122
--- SUPER BLOCK FLAGS ---
Duplicates
Exportable

\end{minted}    
\end{frame}

\begin{frame}[fragile]{Ferramenta de linha de comando: \textit{squashfs-utils}}
Informações da \textit{Inode table}:
   \begin{columns}
   \begin{column}{0.5\textwidth}

\begin{minted}[fontsize=\fontsize{5}{5}, bgcolor=blcodebg]{text}
$./sqfs -i source-dir.sqfs
--- --- ---
{Inode 1/4}
--- --- ---
Permissions: 0x01fd
UID index: 0x0000
GID index: 0x0000
Modified time: Tue 2020-08-04 (yyyy-mm-dd) 15:41:41 CEST
Inode number: 1
Inode type: Basic Directory
Start block: 0x00000000
Hard links: 2
File size: 3
Block offset: 0x0000
Parent inode number: 4

{Inode 2/4}
--- --- ---
Permissions: 0x01b4
UID index: 0x0000
GID index: 0x0000
Modified time: Tue 2020-08-04 (yyyy-mm-dd) 15:42:17 CEST
Inode number: 2
Inode type: Basic File
Start block: 0x00000000
Fragment block index: 0x00000000
Fragment block offset: 0x00000000
(Uncompressed) File size: 12
\end{minted}
   \end{column}
   \begin{column}{0.5\textwidth}
\begin{minted}[fontsize=\fontsize{5}{5}, bgcolor=blcodebg]{text}
{Inode 3/4}
--- --- ---
Permissions: 0x01ff
UID index: 0x0000
GID index: 0x0000
Modified time: Tue 2020-08-04 (yyyy-mm-dd) 15:45:48 CEST
Inode number: 3
Inode type: Basic Symlink
Hard links: 1
Symlink size: 8
Target path: file.txt

{Inode 4/4}
--- --- ---
Permissions: 0x01fd
UID index: 0x0000
GID index: 0x0000
Modified time: Tue 2020-08-04 (yyyy-mm-dd) 15:45:48 CEST
Inode number: 4
Inode type: Basic Directory
Start block: 0x00000000
Hard links: 3
File size: 63
Block offset: 0x0000
Parent inode number: 5
\end{minted}
   \end{column}
   \end{columns}
\end{frame}

\begin{frame}[fragile]{Ferramenta de linha de comando: \textit{squashfs-utils}}
Informações da \textit{Directory table}:

\begin{minted}[fontsize=\fontsize{9}{9}, bgcolor=blcodebg]{text}
$./sqfs -d source-dir.sqfs
Directory 1
Name: dir_example
Empty directory.

Root directory
1) dir_example
2) file.txt
3) link
\end{minted}
\end{frame}

\begin{frame}[fragile]{Ferramenta de linha de comando: \textit{squashfs-utils}}
Recuperando o conteúdo de um arquivo:
\begin{minted}[fontsize=\fontsize{9}{9}, bgcolor=blcodebg]{text}
$ cat source-dir/file.txt
Hello world

$./sqfs -e source-dir.sqfs /file.txt
Hello world
\end{minted}
\end{frame}

\begin{frame}{Migração do código para o U-Boot}
    \begin{figure}
        \centering
        \includegraphics[scale=0.27]{figuras/recycle_transparente.png}
        \caption{Adaptação do código da \textit{squashfs-utils}}
        \label{fig:my_label}
    \end{figure}
\end{frame}

\begin{frame}{Migração do código para o U-Boot}
    \begin{figure}
        \centering
        \includegraphics[scale=0.185]{figuras/uboot.png}
        \caption{Diretório raiz do U-Boot}
        \label{fig:my_label}
    \end{figure}
\end{frame}

\begin{frame}{Implementação da API para sistemas de arquivo do U-Boot}
\begin{figure}
    \centering
    \includegraphics[scale=0.45]{figuras/API.pdf}
    \caption{Funções da API para sistemas de arquivo}
    \label{fig:my_label}
\end{figure}
\end{frame}

\begin{frame}{Implementação da API para sistemas de arquivo do U-Boot}
\begin{figure}
    \centering
    \includegraphics[scale=0.45]{figuras/API2.pdf}
    \caption{Implementação da API para o SquashFS}
    \label{fig:my_label}
\end{figure}
\end{frame}


\begin{frame}[fragile]{Implementando os comandos do U-Boot}

Comando \textit{ls} (\textit{sqfsls}):

\begin{minted}[fontsize=\fontsize{5}{5}, bgcolor=blcodebg]{text}
=> sqfsls 
sqfsls - List files in directory. Default: root (/).

Usage:
sqfsls <interface> [<dev[:part]>] [directory]
    - list files from 'dev' on 'interface' in 'directory'

\end{minted}
\end{frame}

\begin{frame}[fragile]{Implementando os comandos do U-Boot}

Comando \textit{load} (\textit{sqfsload}):

\begin{minted}[fontsize=\fontsize{5}{5}, bgcolor=blcodebg]{text}
=> sqfsload 
sqfsload - load binary file from a SquashFS filesystem

Usage:
sqfsload <interface> [<dev[:part]> [<addr> [<filename> [bytes [pos]]]]]
    - Load binary file 'filename' from 'dev' on 'interface'
      to address 'addr' from SquashFS filesystem.
      'pos' gives the file position to start loading from.
      If 'pos' is omitted, 0 is used. 'pos' requires 'bytes'.
      'bytes' gives the size to load. If 'bytes' is 0 or omitted,
      the load stops on end of file.
      If either 'pos' or 'bytes' are not aligned to
      ARCH_DMA_MINALIGN then a misaligned buffer warning will
      be printed and performance will suffer for the load.
\end{minted}

\end{frame}

\begin{frame}{Implementando os comandos do U-Boot}
\begin{figure}
    \centering
    \includegraphics[scale=0.55]{figuras/uboot-commands.pdf}
    \caption{Diretório raiz do U-Boot}
    \label{fig:my_label}
\end{figure}
\end{frame}

\begin{frame}{Scripts de teste}
    Scripts escritos em Python para testar os comandos do SquashFS: \textit{sqfsload} e \textit{sqfsls}
    
    \begin{figure}
        \centering
        \includegraphics[scale=0.4]{figuras/tests.pdf}
        \caption{Diretório raiz do U-Boot}
        \label{fig:my_label}
    \end{figure}
\end{frame}

\begin{frame}{Submissão do trabalho na mailing list do U-Boot}
   \begin{columns}
   \begin{column}{0.5\textwidth}
\begin{itemize}
\item Consertar todos os problemas de estilo
\item Escrever uma carta de apresentação
\item Dividir o trabalho em \textit{patches}
\end{itemize}
   \end{column}

   \begin{column}{0.5\textwidth}
   Patches:
   \begin{enumerate}
   \item fs/squashfs: new filesystem
   \item cmd/: add filesystem commands
   \item include/u-boot, lib/zlib: add sources for zlib decompression
   \item fs/squashfs: add support for zlib decompression
   \item fs/fs.c: add symbolic link case to fs\_ls\_generic()
   \item test/py: Add tests for the SquashFS commands
   \end{enumerate}
   \end{column}
   \end{columns}
\end{frame}

\begin{frame}{Metodologia de trabalho}
\begin{figure}
    \centering
    \includegraphics[scale=0.5]{figuras/workflow.pdf}
    \caption{Fluxograma da metodologia}
    \label{fig:my_label}
\end{figure}
\end{frame}

\begin{frame}{Metodologia de trabalho}

Execuções do código feitas na \textit{Sandbox} do U-Boot e numa Beagle Bone Black Wireless:

\begin{figure}
    \centering
    \includegraphics[scale=0.3]{figuras/bbb.png}
    \caption{Beagle Bone Black Wireless}
    \label{fig:my_label}
\end{figure}
\end{frame}

\begin{frame}[fragile]{Resultados}
Console do U-Boot, lançado na Beagle Bone Black Wireless:
\begin{minted}[fontsize=\fontsize{4}{4}, bgcolor=blcodebg]{text}
U-Boot SPL 2020.10-rc1-00154-gc7b2d6a45d (Aug 07 2020 - 11:17:02 +0200)
Trying to boot from MMC1

U-Boot 2020.10-rc1-00154-gc7b2d6a45d (Aug 07 2020 - 11:17:02 +0200)

CPU  : AM335X-GP rev 2.1
Model: TI AM335x BeagleBone Black
DRAM:  512 MiB
WDT:   Started with servicing (60s timeout)
NAND:  0 MiB
MMC:   OMAP SD/MMC: 0, OMAP SD/MMC: 1
Loading Environment from FAT... Unable to use mmc 0:1... <ethaddr> not set.
Validating first E-fuse MAC
Net:   Could not get PHY for ethernet@4a100000: addr 0
eth2: ethernet@4a100000, eth3: usb_ether
Hit any key to stop autoboot:  0
=> sqfsls mmc 0:1
            bin/
            boot/
            dev/
            etc/
            lib/
    <SYM>   lib32
    <SYM>   linuxrc
            media/
            mnt/
            opt/
            proc/
            root/
            run/
            sbin/
            sys/
            tmp/
            usr/
            var/

2 file(s), 16 dir(s)
\end{minted}
\end{frame}

\begin{frame}[fragile]{Resultados}
Carregando o kernel e a device tree:
\begin{minted}[fontsize=\fontsize{7}{7}, bgcolor=blcodebg]{text}
=> sqfsload mmc 0:1 $kernel_addr_r /boot/zImage
6091376 bytes read in 476 ms (12.2 MiB/s)
=> sqfsload mmc 0:1 $fdt_addr_r /boot/am335x-boneblack.dtb
40817 bytes read in 14 ms (2.8 MiB/s)
=> setenv bootargs console=ttyO0,115200n8
=> bootz $kernel_addr_r - $fdt_addr_r
## Flattened Device Tree blob at 81000000
   Booting using the fdt blob at 0x81000000
   Loading Device Tree to 8fff3000, end 8fffff70 ... OK

Starting kernel ...

[    0.000000] Booting Linux on physical CPU 0x0
[    0.000000] Linux version 4.19.79 (joaomcosta@joaomcosta-Latitude-E7470)
(gcc version 7.3.1 20180425 [linaro-7.3-2018.05 revision
d29120a424ecfbc167ef90065c0eeb7f91977701] (Linaro GCC 7.3-2018.05))
#1 SMP Fri May 29 18:26:39 CEST 2020
[    0.000000] CPU: ARMv7 Processor [413fc082] revision 2 (ARMv7), cr=10c5387d
\end{minted}
\end{frame}


\begin{frame}{Resultados}
\begin{itemize}
\item O suporte do SquashFS foi aceito e integrado ao código oficial do U-Boot
\begin{figure}
\centering
\includegraphics[scale=0.4]{figuras/tom.jpeg}
\caption{Mensagem de aceite de Tom Rini, administrador principal do U-Boot}
\end{figure}
\item Balanço final: 27 arquivos novos e/ou modificados e aproximadamente 3000 linhas de código
\end{itemize}
\end{frame}



\begin{frame}{Teste2}
xxxxxxxxxx    
\end{frame}
% Resultados
\section{Resultados}
\begin{frame}[fragile]{Resultados}
Console do U-Boot, lançado na Beagle Bone Black Wireless:
\begin{minted}[fontsize=\fontsize{4}{4}, bgcolor=blcodebg]{text}
U-Boot SPL 2020.10-rc1-00154-gc7b2d6a45d (Aug 07 2020 - 11:17:02 +0200)
Trying to boot from MMC1

U-Boot 2020.10-rc1-00154-gc7b2d6a45d (Aug 07 2020 - 11:17:02 +0200)

CPU  : AM335X-GP rev 2.1
Model: TI AM335x BeagleBone Black
DRAM:  512 MiB
WDT:   Started with servicing (60s timeout)
NAND:  0 MiB
MMC:   OMAP SD/MMC: 0, OMAP SD/MMC: 1
Loading Environment from FAT... Unable to use mmc 0:1... <ethaddr> not set.
Validating first E-fuse MAC
Net:   Could not get PHY for ethernet@4a100000: addr 0
eth2: ethernet@4a100000, eth3: usb_ether
Hit any key to stop autoboot:  0
=> sqfsls mmc 0:1
            bin/
            boot/
            dev/
            etc/
            lib/
    <SYM>   lib32
    <SYM>   linuxrc
            media/
            mnt/
            opt/
            proc/
            root/
            run/
            sbin/
            sys/
            tmp/
            usr/
            var/

2 file(s), 16 dir(s)
\end{minted}
\end{frame}

\begin{frame}[fragile]{Resultados}
Carregando o kernel e a device tree:
\begin{minted}[fontsize=\fontsize{7}{7}, bgcolor=blcodebg]{text}
=> sqfsload mmc 0:1 $kernel_addr_r /boot/zImage
6091376 bytes read in 476 ms (12.2 MiB/s)
=> sqfsload mmc 0:1 $fdt_addr_r /boot/am335x-boneblack.dtb
40817 bytes read in 14 ms (2.8 MiB/s)
=> setenv bootargs console=ttyO0,115200n8
=> bootz $kernel_addr_r - $fdt_addr_r
## Flattened Device Tree blob at 81000000
   Booting using the fdt blob at 0x81000000
   Loading Device Tree to 8fff3000, end 8fffff70 ... OK

Starting kernel ...

[    0.000000] Booting Linux on physical CPU 0x0
[    0.000000] Linux version 4.19.79 (joaomcosta@joaomcosta-Latitude-E7470)
(gcc version 7.3.1 20180425 [linaro-7.3-2018.05 revision
d29120a424ecfbc167ef90065c0eeb7f91977701] (Linaro GCC 7.3-2018.05))
#1 SMP Fri May 29 18:26:39 CEST 2020
[    0.000000] CPU: ARMv7 Processor [413fc082] revision 2 (ARMv7), cr=10c5387d
\end{minted}
\end{frame}


\begin{frame}{Resultados}
\begin{itemize}
\item O suporte do SquashFS foi aceito e integrado ao código oficial do U-Boot
\begin{figure}
\centering
\includegraphics[scale=0.4]{figuras/tom.jpeg}
\caption{Mensagem de aceite de Tom Rini, administrador principal do U-Boot}
\end{figure}
\item Balanço final: 27 arquivos novos e/ou modificados e aproximadamente 3000 linhas de código
\end{itemize}
\end{frame}

\begin{frame}{Teste3}
xxxxxxxxxx    
\end{frame}
% Conclusao
\section{Conclusão}
\begin{frame}{Conclusão}
	\begin{itemize}
	    \item Algoritmos de compressão suportados: LZO, GZIP e ZSTD.
	    \item Contribuições recorrentes por parte da comunidade (internacional) do U-Boot
	    \item Otimizações de seções do código
	    \item Contribuições paralelas à documentação não-oficial usada como referência
	\end{itemize} 
\end{frame}
\begin{frame}{Teste}
xxxxxxxxxx    
\end{frame}
% Referencias
\include{tex/referencias/referencias}
\begin{frame}{Teste}
xxxxxxxxxx    
\end{frame}
% Agradecimentos
\section{}
\begin{frame}{Agradecimentos}
	Agradeço a todos. 	
\end{frame}

\end{document}